\chapter{Backend}\label{ch:backend}
Having the program source parsed and delivered through the derivation tree elements, we now
concentrate on the compiler backend.
In this chapter, we begin with handling the symbol tables and defining the data structures, in Section
\ref{sec:adjustments-to-derivation-tree-element-(dte)-model}, we extend the \verb+DTE+
with helper methods (defined in the system architecture book).
Section~\ref{sec:using-new-tools-for-filling-the-symbol-tables} focuses on integrating the Sethi-Ulman algorithm,
creating the debugging tools and the context utilities (current function getter, function call initializer etc).
Finally, in Section~\ref{sec:code_gen} we describe the whole process of the code generation from the given derivation tree.
In Section~\ref{sec:translated_programs} we provide example programs with the translated versions.


\section{Symbol Table Data Structures}\label{sec:symbol_table_models}
In this section, we provide definitions of the data structures the for symbol tables, specifically the type table, variable table, and function table.\
These tables are essential for accessing pertinent program data prior to code generation.
Example usages include retrieving function parameters by name, searching for struct components, ensuring type safety, calculating displacements etc.
The implementation of the data structures can be found in packages \verb+model+ and \verb+table+.

The base models are as follows
\begin{enumerate}
    \item \verb+VarType+ - Refers to the variable's type, containing type-specific data such as type class, name,
    size, struct components, pointer target type and array target type.
    \item \verb+Variable+ - Stores the variable's name, type (\verb+VarType+) and displacement.
    \item \verb+Fun+ - Stores the function's name, return type, memory struct (parameters combined with local variables) and the function's body.
\end{enumerate}

\subsection{VarType}\label{subsec:vartype}
\textbf{Attributes:}
\begin{itemize}
    \item \verb+typeClass: TypeClass+ - an enum indicating whether the type is primitive (int, uint, char, bool) or constructed with
    a type definition (array, pointer, struct).
    \item \verb+name: string+ - referring to the type's name.
    \item \verb+size: int+ - size of the type in bytes.
    \item \verb+pointerTypeTargetName: string+ - name of the pointer's target type, equals to \verb+null+\\ if \verb+typeClass != POINTER+
    \item \verb+arrayCompTypeTargetName: string+ - name of the array's component type, equals to \verb+null+ if \verb+typeClass != ARRAY+
    \item \verb+arraySize: int+ - number of elements of the array, equals to 0 if \verb+typeClass != ARRAY+
    \item \verb+structComponents: Map<String, Variable>+ - components of the struct, equals to \verb+null+\\ if \verb+typeClass != STRUCT+
\end{itemize}

\subsection{Variable}\label{subsec:variable}
\textbf{Attributes:}
\begin{itemize}
    \item \verb+name: string+ - The variable's name.
    \item \verb+type: VarType+ - The variable's type.
    \item \verb+baseAddress: int+ - In case of global memory struct \verb+gm+, refers to the base pointer BPT,\\ 0 - otherwise.
    \item \verb+displacement: int+ - Refers to the displacement from the parent memory struct.
\end{itemize}

\subsection{Fun}\label{subsec:fun}
\textbf{Attributes:}
\begin{itemize}
    \item \verb+name: string+ - The function's name.
    \item \verb+returnType: VarType+ - Function's return type.
    \item \verb+memoryStruct: Variable+ - Refers to the variable of type class struct, storing as components the parameters
    and local variables of the function.
    \item \verb+numParameters: int+ - needed to differentiate between parameters and local variables in \verb+memoryStruct+
    \item \verb+body: DTE+ - Subtree of the function's body.
\end{itemize}

\subsection{Tables}\label{subsec:tables}
For representing the tables of the aforementioned models, we introduce the following interface:
\begin{codeblock}[Table. Java implementation on \href{https://github.com/fyfsb/dcfg/blob/main/src/main/java/table/Table.java}{Github}]
    interface Table {
        void fillTable(DTE content) throws Exception;
    }
\end{codeblock}
Respective implementations of the interface are:
\begin{enumerate}
    \item \verb+TypeTable+ - contains \verb+Map<String, VarType>+ mapping for the types.
    Provides simplified methods for creating types and/or adding to type tables by the type class.
    \item \verb+MemoryTable+ - contains \verb+Map<String, Variable>+ mapping (only for the variables with struct type class).
    Overridden method \verb+fillTable+ creates global memory struct.
    \item \verb+FunTable+ - contains \verb+Map<String, Fun>+ mapping.
\end{enumerate}


\section{Adjustments to Derivation Tree Element (DTE)}\label{sec:adjustments-to-derivation-tree-element-(dte)-model}
In the previous chapter, the class for the derivation tree element was introduced.
For the code generation, we extend the class with new methods that help to extract meaningful data from the subtree.
As an example, we show the derivation tree of the following variable declaration sequence:\\ \verb+int x; char y+ (Figure \ref{fig:example_subtree})
\begin{figure}[h]
    \begin{codeblock}[Derivation tree element. Java implementation on \href{https://github.com/fyfsb/dcfg/blob/main/src/main/java/tree/DTE.java}{Github}]
class DTE {
    Symbol label;
    DTE father;
    DTE fson;
    DTE bro;
}
    \end{codeblock}\label{fig:figure2}

\end{figure}

\begin{figure}[h]
    \centering
    \begin{tikzpicture}
        [font=\small,sibling distance =1.5cm, grow'=up, edge from parent/.style={draw=black, thick,->},
        every node/.style={style A}]

        \node (root) {VaDS}

        child {
            node (VaDS) {VaDS}
            child {
                node (VaD2) {VaD}
                child {
                    node (Ty1) {Ty}
                    child { node[blue] (int) {int} }
                }
                child {
                    node (Na1) {Na}
                    edge from parent[draw=none]
                    child { node (Le1) {Le} child {node[blue] (x) {x} } }
                }
            }
        }
        child {
            node[blue] (sc) {;}
            edge from parent[draw=none]
        }
        child {
            node (VaD1) {VaD}
            edge from parent[draw=none]
            child { node (Ty2) {Ty} child { node[blue] (char) {char}} }
            child {
                node (Na2) {Na}
                edge from parent[draw=none]
                child { node (Le2) {Le} child { node[blue] (y) {y} }}
            }
        };

        \draw[thick, ->] (VaDS)  -- (sc);
        \draw[thick, ->] (sc)    -- (VaD1);
        \draw[thick, ->] (Ty1)   -- (Na1);
        \draw[thick, ->] (Ty2)   -- (Na2);
    \end{tikzpicture}
    \caption{Example subtree}
    \label{fig:example_subtree}
\end{figure}

\subsection{Flattened Sequence}\label{subsec:$fseq$}
The population of symbol tables and subsequent code generation is accomplished by thoroughly parsing the program tree.
Certain grammar tokens have the corresponding sequence element that has the derivation exemplified by
\[XS \to X\ |\ XS; X\]
In the context of symbol definitions, there is $\langle TyDS\rangle, \langle VaDS \rangle$ and $\langle FuDS \rangle$.
To optimize the input for table models, an array of the respective tree elements is preferred over a singular sequence element.
To facilitate this, a method for flattening the sequence is introduced.
\begin{definition}[fseq]
    Let $t = XS$ be a sequence subtree, i.e.
    \[XS \to X \ |\ XS;X\qquad X,XS\in N\]
    Then the flattened sequence of the subtree is given by
    \[fseq(t) = \left[X_1,\dots,X_k\right]\]
    where $k$ is the amount of derived $X$ non-terminals in the tree.
    Algorithm is implemented in the following way:
\end{definition}

\begin{codeblock}[Flattened sequence. Java implementation on \href{https://github.com/fyfsb/dcfg/blob/main/src/main/java/tree/DTE.java}{Github}]
    input: sequence subtree xs,
    output: flattened sequence DTE[]

    fseq(DTE xs) -> DTE[] {
        case xs.label is undefined -> return [];
        case xs.fson.bro is undefined -> [xs.fson];
        otherwise -> fseq(xs.fson) + [xs.fson.bro.bro];
    }
\end{codeblock}
First case is a simple null check, following 2 describe
possible derivations of $xs$:
\begin{itemize}
    \item \verb+xs.fson.bro is undefined+: sequence element derives $XS \to X$. \verb+xs.fson+ is the only
    element in the sequence.
    \item \verb+otherwise+: $xs$ derives $XS \to XS;X$.
    Flattened sequence is recursively obtained for \verb+xs.fson+, to which \verb+xs.fson.bro.bro+ is appended
    (Corresponding to $X$ in the derivation)
\end{itemize}
Working with example from \ref{fig:example_subtree}, we have
\[fseq(VaDS) = [VaD(\verb+int+, \verb+x+), VaD(\verb+char+, \verb+y+)]\]

\subsection{Border Word}\label{subsec:border-word}
Symbol table entries have string values as keys, extracting the name from the respective symbol definition/declaration
requires traversing and concatenating the labels of the leaves of the subtree \verb+<Na>+.
For this purpose, function getting the border word of the subtree is used:
\begin{definition}[bw]
    Let $x \in N \cup T$ be a root for the program subtree.
    Border word of the subtree is given by
    \[
        bw(\verb+x+) = \begin{cases}
                           \verb+x.label.content+&\quad x\in T\\
                           bw(\verb+x.fson+)\circ \bigcirc_{i=1}^k bw(\verb+x.bro+_i)&\quad x\in N
        \end{cases}
    \]
    where $k$ is the number of siblings of \verb+x.fson+\\
    Algorithm:
\end{definition}
\begin{codeblock}[Border word. Java implementation on \href{https://github.com/fyfsb/dcfg/blob/main/src/main/java/tree/DTE.java}{Github}]
    input: derivation tree element x.
    output: concatenated labels of the leaves, string res.

    bw(DTE x) -> string {
        case x is terminal -> return x.label.content;
        otherwise -> {
            res := bw(x.fson);
            for all siblings bro_i of x.fson -> {
                res = res + bw(bro_i);
            }
            return res;
        }
    }
\end{codeblock}

\subsection{Extract Component Pairs}\label{subsec:extract-component-pairs}
The current implementation of the filling of the symbol tables requires extracting the pairs
of form \verb+Ty Na+ (variable declaration) in the string form and performing the lookups in the
type table and the memory struct, respectively.
This step ensures that the context conditions (type being defined, variable name being unique) are satisfied.
We're referring to the aforementioned pattern as a pair component, i.e a pair of tokens $A, B$ derived as
\[ XS \to XS; X\ |\ X\qquad XS,X \in N\qquad A,B\in N\cup T\]
One clear example of such pattern is variable (or parameter) declaration sequence.
\[ VaD \to Ty\ Na\]
Reading the variable declarations occurs at all 3 stages of filling the symbol tables:
\begin{itemize}
    \item Type definitions: reading the components of struct type;
    \item Global variable declarations: consists only of variable declarations;
    \item Function declarations: parameters and local variables.
\end{itemize}
An attempt at simplifying the implementation entails extracting the component pair sequence reads
into the following algorithm:

\begin{definition}[extractCompPairs]
    Let $X \in N$ be a non-terminal deriving a pair
    \[X \to A\ B\]
    where $A,B \in N\cup T$.
    Let $XS \in N$ be a sequence non-terminal deriving
    \[ XS \to XS;X\ |\ X\]
    then extraction of all the component pairs yields the list of tuples of the form
    \[(bw(\verb+A+), bw(\verb+B+)) \]
    We have
    \[ \verb+extrCompPairs(XS)+ = \Bigg[ \Big( bw(fs_1.fson),bw(fs_1.fson.bro) \Big),\dots,\Big( bw(fs_k.fson),bw(fs_k.fson.bro) \Big) \Bigg]\]
    where for $i\in[1:k],\ fs_i \in fseq(xs),\ k=|fseq(xs)|$
\end{definition}
\begin{codeblock}[Extracting component pairs. Java implementation on \href{https://github.com/fyfsb/dcfg/blob/main/src/main/java/tree/DTE.java}{Github}]
    input: DTE xs - sequence element deriving component pairs parent x
    output: string[][] - array of pair border word tuples

    extrCompPairs(DTE xs) -> string[][] {
        fs := fseq(xs)
        result := []
        for all elements fs_i in fs -> {
            case fs_i is separator token (";", ",") -> fs_i = fs_i.bro;

            A := bw(fs_i.fson)
            B := bw(fs_i.fson.bro)

            result.add([A,B])
        }
        return result
    }
\end{codeblock}
Aforementioned function, for the example from Figure~\ref{fig:example_subtree} would return
\[extrCompPairs(VaDS) = [["int", "x"], ["char", "y"]]\]


\section{Filling The Symbol Tables}\label{sec:using-new-tools-for-filling-the-symbol-tables}
In this section, we present a comprehensive overview of the symbol table filling process, leveraging the predefined models and algorithms.
Our approach is guided by the following strategies:
\begin{enumerate}
    \item Type tables:
    \begin{itemize}
        \item Arrays: The array type must be either primitive or defined beforehand.
        Additionally, the array size is validated for correctness, and the new type name should not already exist in the type table.

        \item Structs: Same constraints apply to each component of the structure.
        \item Pointers: It is not required for the pointer target type to be defined previously.
        For that reason, \verb+VarType+ model stores only the name of the target, obtaining the type instance
        requires a lookup in the table.
    \end{itemize}
    \item Global variable tables:
    \begin{itemize}
        \item Variable names should be distinct.
        \item Types should be either primitive, or previously defined.
    \end{itemize}
    \item Function tables:
    \begin{itemize}
        \item Names should be distinct (in this version, we do not account for signatures.
        Thus, function overloading is not allowed).
        \item Return types should be primitive or defined previously.
        \item Parameters and local variables follow the same constraints as global variables.
    \end{itemize}
\end{enumerate}
General overview of the compilation process is shown in Figure~\ref{fig:c0_comp}.
The program tree derives 3 sequence tokens \verb+<TyDS>+, \verb+<VaDS+ and \verb+FuDS+.
Subsequently, the corresponding tables are populated.
Once completed, the body of the main function is extracted from the function table.
For each statement within the main function, assembly code is then generated.
\begin{figure}[h]
    \centering
    \begin{tikzpicture}
        [font=\small,sibling distance =1.5cm, level distance=3cm, grow'=right, edge from parent/.style={draw=black, thick,->},
        level 4/.style={level distance=3cm},
        level 5/.style={grow'=down, level distance=1.5cm}]


        \node[style A] (root) {C0 code}

        child {
            node[style B] (lp) {Lexer + Parser}
            child {
                node[style A] (pt) {Program tree}
                child {
                    node[style A] (tyds) {TyDS}
                    child {
                        node[style B] (tt) {TypeTable}
                    }
                }
                child {
                    node[style A] (vads) {VaDS}
                    child {
                        node[style B] (mt) {MemoryTable}
                    }
                }
                child {
                    node[style A] (fuds) {FuDS}
                    child {
                        node[style B] (ft) {FunctionTable}
                        child {
                            node[style A] (body) {\$main.body}
                            child {
                                node[style B] (cg) {CodeGenerator}
                                child { node[style C] (ai) {assembly instructions} }
                            }
                        }
                    }
                }
            }
        };

    \end{tikzpicture}\caption{C0 Program Compilation Process}
    \label{fig:c0_comp}
\end{figure}

\subsection{Filling the type table}\label{subsec:filling-the-type-table}
Filling the type table from the \verb+<TyDS>+ token involves flattening the sequence and reading each
type definition separately (Figure~\ref{fig:fill_tt}).
\begin{figure}[h]
    \centering
    \begin{tikzpicture}
        [font=\small,sibling distance =1.5cm, level distance=3cm, grow'=right, edge from parent/.style={draw=black, thick,->},
        level 3/.style={level distance=1.2cm},
        level 5/.style={grow'=down, level distance=1.5cm}]

        \node[style B]  at (7.2,-2) (readTyD) {readTyD};
        \node[style A, minimum width=4cm, minimum height=1cm] at (7.2, 0) (box) {};
        \node[style A] (tyds) {TyDS}
        child {
            node[style B] (fseq) {\ fseq\ }
            child {
                node[style A] (tyd1) {TyD1} edge from parent[draw=none]
                child {
                    node[draw=none] (dots) {\dots} edge from parent[draw=none]
                    child {
                        node[style A] (tydn) {TyDn} edge from parent[draw=none]
                    }
                }
            }
        };

        \draw[thick, ->] (box)  -- (readTyD);
        \draw[thick, ->] (fseq)  -- (box);
    \end{tikzpicture}\caption{Filling the type table}
    \label{fig:fill_tt}
\end{figure}
\newpage
Each time a derivation tree element is processed, the corresponding pattern matching function ensures
that the passed argument satisfies the required label, as exemplified below.
\begin{codeblock}[fillTypeTable. Java implementation on \href{https://github.com/fyfsb/dcfg/blob/main/src/main/java/table/TypeTable.java}{Github}]
    input: DTE tyds - tree element of type TyDS,
    output: none
    effect: fills the table with new types

    fillTypeTable(DTE tyds) -> void {
        case tyds is not <TyDS> -> error;
        for all element tyD in fseq(tyds) -> readTyD(tyD);
    }
\end{codeblock}
Next step involves implementing \verb+readTyD+.
Given the derivation for \verb+TyD+ tokens, as shown in\\ Figure~\ref{fig:tyd}, uniqueness of the type's name is checked as
\begin{codeblock}
    checkUnique(DTE tyD) {
        nameToken := tyD.fson.bro.bro
        name := bw(nameToken)
        if TypeTable.containsType(name) -> { error }
    }
\end{codeblock}
Subsequently, we perform a case split based on the kind of the
type expression, illustrated in Figure~\ref{fig:readTyD}.
The \verb+struct+ keyword identifies the struct class, the asterisk at the end indicates a pointer, and opening square brackets denote arrays.
These key distinctions are crucial, as the actual implementation (unlike the representation in Figure~\ref{fig:readTyD}) relies on pattern matching solely based on these differences.
Consequently, visiting the first two children of the type expression subtree is sufficient.
\begin{figure}[h]
    \centering
    \begin{tikzpicture}
        [font=\small, level distance=2cm, grow'=up, edge from parent/.style={draw=black, thick,->},
        level 2/.style={grow'=right}]

        \node[style A] (tyd) {TyD}
        child {
            node[style A] (td) {typedef}
            child {
                node[style A] (te) {TE}
                child { node[style A] (na) {Na} }
            }
        };

    \end{tikzpicture}\caption{Derivation of the type definition}
    \label{fig:tyd}
\end{figure}

\begin{figure}[h]
    \centering
    \begin{tikzpicture}
        [font=\small, level distance=2cm, grow'=right, edge from parent/.style={draw=black, thick,->},
        level 2/.style={grow'=right, level distance=1.1cm}]

        \node[style B]  at (8,1.5) (createArray) {createArray};
        \node[style B]  at (8,0) (createPointer) {createPointer};
        \node[style B]  at (8,-1.5) (createStruct) {createStruct};

        \node[style A] (te) {TE}
        child {
            node[style A] (ty1) {Ty} edge from parent[draw=none]
            child {
                node[style A] (lb) {[}
                child {
                    node[style A] (dis) {DiS}
                    child {
                        node[style A] (rb) {]}
                    }
                }
            }
        }
        child {
            node[style A] (ty2) {Ty} edge from parent[draw=none]
            child { node[style A] (ptr) {$*$} }
        }
        child {
            node[style A] (struct) {struct} edge from parent[draw=none]
            child {
                node[style A] (lcb) {\{}
                child {
                    node[style A] (vads) {VaDS}
                    child { node[style A] (rcb) {\}} }
                }
            }
        };

        \draw[thick, ->] (rb)  -- (createArray);
        \draw[thick, ->] (ptr)  -- (createPointer);
        \draw[thick, ->] (rcb)  -- (createStruct);


        \draw[dotted,thick, ->] (te)  -- (ty1);
        \draw[dotted,thick, ->] (te)  -- (ty2);
        \draw[dotted,thick, ->] (te)  -- (struct);
    \end{tikzpicture}\caption{Reading the type definition}
    \label{fig:readTyD}
\end{figure}

\newpage

\subsubsection{Adding array type}
Figure~\ref{fig:addArray} illustrates the process of adding an array type to the type table.
Initially, the \verb+<Ty>+ type is checked for being primitive or defined previously.
This is achieved by extracting the type name from the border word of the type token and searching for the corresponding \verb+VarType+ entry in the type table.
\begin{codeblock}
    checkDefined(DTE ty) -> bool {
        name := bw(ty)
        return TypeTable.containsType(ty)
    }
\end{codeblock}
The subsequent step involves verifying the non-negativity of array's length.
Any failure in these checks results in an error.
Upon passing the check, the array's target type is obtained from the corresponding entry in the type table.
The array's size is extracted from the parsed \verb+<DiS>+ token, and name is the border word of \verb+<Na>+.
The resulting record is then added to the type table.

\begin{figure}[h]
    \centering
    \begin{tikzpicture}
        [font=\small, level distance=3.1cm, grow'=right, edge from parent/.style={draw=black, thick,->},
        level 2/.style={grow'=right, level distance=1.1cm}]

        \node[style B] at (2, -8.5) (addToTT) {addToTypeTable};
        \node[style A] at (2, -6) (done) {
            \begin{minipage}{5cm}
                \begin{verbatim}arrayTargetTy := tt(Ty);
arraySize := parse(dis);
name := bw(na)
                \end{verbatim}
            \end{minipage}
        };
        \node[style B] at (1,-4) (valid) {checkValidNumber};
        \node[style B] at (0, -2) (defined) {checkDefined}
        child { node[style D] (err) {Error} edge from parent  [->] node [above] {\tiny false}};
        \node[style A] (ty1) {Ty}
        child {
            node[style A] (lb) {[}
            child {
                node[style A] (dis) {DiS}
                child {
                    node[style A] (rb) {]}
                }
            }
        };
        \draw[thick, ->] (ty1) -- (defined);
        \draw[thick, ->] (defined) -- node[midway, left] {\tiny true} (valid);
        \draw[thick, ->] (valid) -- node[midway, right] {\tiny false} (err);
        \draw[thick, ->] (dis) .. controls (4,-4) .. node[midway, right] {dis} (valid);
        \draw[thick, ->] (valid) -- node[right] {\tiny true} (done);
        \draw[thick, ->] (done) -- node[right] {\tiny result} (addToTT);
    \end{tikzpicture}\caption{Adding array type}
    \label{fig:addArray}
\end{figure}
\newpage

\subsubsection{Adding Pointer Type}
As described in the beginning, strategy for adding the pointers doesn't require the target type to be defined
previously, thus possibly resulting in a NULL-type pointer (initially).
Actual access for the target type involves lookups in the type table:
\begin{codeblock}
    getPointerTargetType() -> VarType {
        type := TypeTable.getTypeOrNull(ptrTargetTyName)
        if type == null -> { error }
        return type
    }
\end{codeblock}
\begin{figure}[h]
    \centering
    \begin{tikzpicture}
        [font=\small, level distance=2.1cm, grow'=down, edge from parent/.style={draw=black, thick,->},]
        \node[style A] (done) {
            \begin{minipage}{5.5cm}
                \begin{verbatim}ptrTargetTy = tt(Ty) || NULL;
name := bw(na);
                \end{verbatim}
            \end{minipage}
        }
        child { node[style B] {addToTypeTable}};
    \end{tikzpicture}\caption{Adding pointer type}
    \label{fig:addPointer}
\end{figure}

\subsubsection{Adding Struct type}
In Figure~\ref{fig:addStruct}, the checking process for structs components is illustrated.
Similar to arrays, the type being primitive or defined previously is verified using the same checker function.
However, uniqueness of the names is required within the components.
To address this, a set is initialized and updated within the loop.
Additionally, inside the loop, displacement is accumulated, satisfying the formula
\[ displ(comp_i, struct) = ba(struct) + \sum_{j=0}^{i-1}size(comp_j)\]
At the end of the loop, the displacement variable represents the sum of the sizes of all components,
equivalent to the size of the entire struct, which is then assigned to the respective attribute.
After passing the name, size and components to the struct type builder, new instance of the \verb+VarType+
is added to the type table.
\begin{codeblock}[Adding struct type. Java implementation on \href{https://github.com/fyfsb/dcfg/blob/main/src/main/java/table/TypeTable.java}{Github}]
    ...
    components := fseq(VaDS)
    compNames := emptySet()
    strComps := Map<String, Variable>()
    displacement := 0
    for all comp in components -> {
        ty := comp.fson
        na := comp.fson.bro

        checkDefined(ty)
        if compNames.contains(na) -> { error }
        compNames.add(na)

        type := TypeTable.getType(bw(ty))
        strComps.put(na, new Variable(na, 0, type, displacement))
        displacement += size(type)
    }
    structType := structBuilder
        .setName(bw(Na))
        .setSize(displacement)
        .setComps(strComps).build()
    TypeTable.addType(structType)
\end{codeblock}
\begin{figure}[h]
    \centering
    \begin{tikzpicture}
        [font=\small, level distance=1.5cm, grow'=right, edge from parent/.style={draw=black, thick,->},
        level 2/.style={grow'=right, level distance=1.5cm}]

        \node[style B] at (3,-9) (addToTT) {addToTypeTable};
        \node[style A] at (3,-7) (code) {
            \begin{minipage}{8cm}
                \begin{verbatim}structComps := mapping(VaDi.Na, tt(VaDi));
name := bw(na)
                \end{verbatim}
            \end{minipage}
        };
        \node[style D] at (7, -5) (err) {Error};
        \node[style B] at (3, -5) (check) {\verb+checkDefinedAndNameUnique+};
        \node[style A, minimum width=4.5cm, minimum height=1cm] at (3, -3) (box) {};
        \node[style A] at (1.5,-3) (vad1) {VaD1}
        child {
            node[draw=none] (dots) {$\dots$} edge from parent[draw=none]
            child { node[style A] (vadn) {VaDn} edge from parent[draw=none]}
        };
        \node[style B] at (3,-1.5) (fseq) {fseq};
        \node[style A] (struct) {struct}
        child {
            node[style A] (lcb) {\{}
            child {
                node[style A] (vads) {VaDS}
                child { node[style A] (rcb) {\}} }
            }
        };

        \draw[thick, ->] (vads) -- (fseq);
        \draw[thick, ->] (fseq) -- (box);
        \draw[thick, ->] (box) -- node[midway, right] {for each} (check);
        \draw[thick, ->] (check) -- node[midway, above] {false} (err);
        \draw[thick, ->] (check) -- node[midway, right] {all true} (code);
        \draw[thick, ->] (code) -- (addToTT);
    \end{tikzpicture}\caption{Adding Struct type}
    \label{fig:addStruct}
\end{figure}
\newpage

\subsection{Filling memory table}\label{subsec:filling-memory-table}
Since variable scopes are decided to be stored inside a struct, we treat global
variable declaration sequence \textbf{almost} as a type definition of form:

\begin{codeblock}
typedef struct { VaDS } $gm;
\end{codeblock}
Single difference from a regular type definition is, that the newly created type
is not added to the type table, otherwise the following code would be a valid C0 program:
\begin{codeblock}
int a;
bool b;
char c;
int main() {
    $gm k;
    ...
    k.b = true;
    return k.a
}
\end{codeblock}
Therefore, we're employing the modified function for adding the struct types:
\newpage
\begin{codeblock}
...
structType := structBuilder.setName(bw(Na)).setSize(displacement).setComps(strComps).build()
return structType // just creating and returning, without adding to the type table
\end{codeblock}
\subsection{Filling the function table}
Approach is similar to the type tables, as shown in Figure~\ref{fig:fillFT}, function definitions are flattened and
read separately.
\begin{figure}[h]
\centering
\begin{tikzpicture}
[font=\small,sibling distance =1.5cm, level distance=3cm, grow'=right, edge from parent/.style={draw=black, thick,->},
level 3/.style={level distance=1.2cm},
level 5/.style={grow'=down, level distance=1.5cm}]

\node[style B]  at (7.2,-2) (readTyD) {readFuD};
\node[style A, minimum width=4cm, minimum height=1cm] at (7.2, 0) (box) {};
\node[style A] (tyds) {FuDS}
child {
node[style B] (fseq) {\ fseq\ }
child {
node[style A] (tyd1) {FuD1} edge from parent[draw=none]
child {
node[draw=none] (dots) {\dots} edge from parent[draw=none]
child {
node[style A] (tydn) {FuDn} edge from parent[draw=none]
}
}
}
};

\draw[thick, ->] (box)  -- (readTyD);
\draw[thick, ->] (fseq)  -- (box);
\end{tikzpicture}
\caption{Filling the function table}
\label{fig:fillFT}
\end{figure}

As in the previous parts, type is verified to be either primitive or defined previously. Upon success, return
type is stored:
\begin{codeblock}
ty := bw(Ty)
checkDefined(ty)
returnType = TypeTable.getType(ty)
\end{codeblock}
function name is checked for uniqueness and the respective attribute assigned:
\begin{codeblock}
na := bw(Na)
if FunctionTable.containsFunction(na) -> { error }
name := na
\end{codeblock}
A local memory struct is generated by merging the sequences of parameters and local variables.
Initially, the number of parameters is stored. As shown in Figure~\ref{fig:readFuD},
\verb+<PaDS>+ and \verb+<VaDS>+ are flattened and combined into a single list. Under the hood, component pairs
are extracted and forwarded to the struct builder (Utilizing the builder pattern for constructing \verb+VarType+ instances,
the project's source code provides a reference for this implementation).
\begin{codeblock}
componentPairs := extractCompPairs(PaDS)
componentPairs.addAll(extractCompPairs(VaDS))
structBuilder.setCompPairs(componentPairs)
... // same logic follows, as in creation of the global memory struct
\end{codeblock}
\begin{figure}[h]
\centering
\begin{tikzpicture}
[font=\small,sibling distance =1.5cm, level distance=1.5cm, grow'=right, edge from parent/.style={draw=black, thick,->}]

\node[style A] at (12, -3) (addbody) {\verb+setBody+};
\node[style A] at (8.02, -6) (struct) {
\begin{minipage}{5cm}
\begin{verbatim}
// same as in $gm
create memstruct for [1,n+m]
memStruct := $bw(na)
\end{verbatim}
\end{minipage}
};
\node[style A, minimum width=6cm, minimum height=1cm] at (8.02, -4.5) (box) {};
\node[style A] at (5.7,-4.5) (vad1) {$VaD_1$};
\node[draw=none] at (6.8, -4.5) (dots1) {$\dots$};
\node[style A] at (7.9,-4.5) (vadn) {$VaD_n$};
\node[draw=none] at (9, -4.5) (dots2) {$\dots$};
\node[style A] at (10.1,-4.5) (vadnm) {$VaD_{n+m}$};
\node[style B] at (7, -3) (fseq) {\verb+fseq+};
\node[style A] at (3, -7.5) (nameCode) {
\begin{minipage}{2.5cm}
\begin{verbatim}
name := bw(na);
\end{verbatim}
\end{minipage}
};
\node[style B] at (3, -6) (functionExists) {\verb+functionExists+};
\node[style A] at (0, -4.5) (checkCode) {
\begin{minipage}{4cm}
\begin{verbatim}
returnType = tt(bw(ty))
\end{verbatim}
\end{minipage}
};
\node[style D] at (3, -3) (err)    {Error};
\node[style B] at (0, -3) (check) {\verb+checkDefined+};
\node[style A] (fud) {\verb+FuD+};
\node[style A] at (0, -1) (ty) {Ty}
child {
node[style A] (na) {Na}
child {
node[style A] (lp) {(}
child {
node[style A] (pads) {PaDS}
child {
node[style A] (rp) {)}
child {
node[style A] (lcb) {\{}
child {
node[style A] (vads) {VaDS}
child {
node[style A] (sc) {;}
child {
node[style A] (body) {body}
child{
node[style A] (rcb) {\}}
}
}
}
}
}
}
}
}
};
\draw[thick, ->] (fud) -- (ty);
\draw[thick, ->] (ty) -- node[midway, left] {$tt(bw(ty))$} (check);
\draw[thick, ->] (check) -- node[midway, above] {false} (err);
\draw[thick, ->] (check) -- node[midway, right] {true} (checkCode);
\draw[thick, ->] (checkCode) -- (functionExists);
\draw[thick, ->] (na) .. controls (5, -3) .. node[midway,right] {$bw(na)$} (functionExists);
\draw[thick, ->] (functionExists) -- node[midway, right] {true} (err);
\draw[thick, ->] (functionExists) -- node[midway, left] {false} (nameCode);
\draw[thick, ->] (pads) -- (fseq);
\draw[thick, ->] (vads) -- (fseq);
\draw[thick, ->] (fseq) -- (box);
\draw[thick, ->] (box) -- (struct);
\draw[thick, ->] (body) -- (addbody);
\end{tikzpicture}
\caption{Reading function definition}
\label{fig:readFuD}
\end{figure}
In the final step, a new function entry is added to the table. With the symbol tables now filled, we proceed to the implementation
of code generation.

\section{Helper Components for Code Generation}\label{sec:helpers}
In the upcoming section, we lay the groundwork for our code generation process by introducing crucial
helper components. Among these, we delve into the assignment of registers using pebble game. Furthermore, we
introduce the \verb+FunctionCall+ class, designed to streamline the management of function calls. This class stores
essential information such as the function, recursion depth, and displacement from the base pointer. Besides the practical
purpose, \verb+FunctionCall+ enables convenient debugging. Additionally, the \verb+Configuration+ class is implemented, that
not only manages the stack of function calls, but also provides access to the current function and incorporates the pebble game
logic, including acquiring free registers and releasing them when needed. Finally, we introduce new record \verb+VarReg+, which is
simply a data class containing the variable, type and the register. Later on, storing the register inside a record enables
dividing the code generation parts by context and allows working with repeating patterns recursively.
\subsection{Integrating Pebble Game}
Referencing the Sethi-Ulman algorithm described in the book, we align the pebble game moves with the state of registers in the statement,
as outlined below:
\begin{itemize}
    \item A register is considered occupied when a pebble is placed on a corresponding node.
    \item Moving the pebble doesn't alter the number of free registers, an occupied register is retained.
    \item Removing the pebble from the node releases the associated register, rendering it available.
    \item All registers are freed in the end of the statement.
\end{itemize}
Therefore, a register is assigned for the expression node, it is released either when the statement is completed, or after
the binary operation if the respective pebble, associated with one of the operands, is removed from the node, i.e. is not
storing the result of the operation. Given that, we're choosing an approach different from the pebble game. The main idea is
to keep track of the occupied registers, release the single register corresponding to the binary operation's right operand, after the operation's finished,
and release every register after each statement.
\subsubsection{Keeping the track of the occupied registers with a single pointer}
Key idea is to employ a straightforward pointer to the first available register, occupying the register entails increasing the pointer,
releasing triggers the decrement.
\begin{codeblock}
int registerPointer := 1; // first free register

getFreeReg() -> int {
    if registerPointer >= MAX_AVAILABLE -> { error }
    registerPointer := registerPointer + 1
    return registerPointer - 1
}

releaseReg() {
    registerPointer = min(1, registerPointer - 1)
}
\end{codeblock}
Although this implementation satisfies the current grammar, the naive approach does not work for the extended grammar
with inline assembly portions:
\begin{codeblock}[Inline assembly]
gpr(j) = e { 4, 6 };
e = gpr(j) { 2, 7 };
// Here, e is an expression
\end{codeblock}

The aforementioned listing is an example of the inline assembly statements of C0. The sets of numbers right after the assignments
specify the registers which are restricted to be changed during the respective assignment. Although,
simple repeated incrementing of the register pointer will work while evaluating inline assembly statements, although releasing the
registers will result in a single decrement, leaving out a range of available registers out of scope.

\subsubsection{Pebble Game With Boolean Arrays}
Workaround is to use a boolean array, with each item indicating the availability of the register
with the respective index. Getting the register becomes an operation of getting first one and setting zero,
releasing - setting one. This approach works for inline assembly portions as well, since it suffices to make
each register from set \verb+J+ unavailable during the statement.
\begin{codeblock}[Pebble game]
registers = new boolean[MAX_AVAILABLE];
...
int reg = f1(registers); // getting first one
registers[reg - 1] = false; // occupying while in use
...
registers[reg - 1] = true; // releasing the register
\end{codeblock}
\subsection{Configuration Class}
It is important to keep track of the current function (for variable binding). For debugging, it's also helpful to
store the recursion depth and the displacement of each call. We define a record for this purpose
\begin{codeblock}[Function call. Java implementation on \href{https://github.com/fyfsb/dcfg/blob/main/src/main/java/config/FunctionCall.java}{Github}]
class FunctionCall {
    Fun function;
    int rd;
    int displacement; // from bpt
}
\end{codeblock}
We're combining the function call stack and pebble game logic into Configuration class.
\begin{codeblock}[Configuration class. Java implementation on \href{https://github.com/fyfsb/dcfg/blob/main/src/main/java/config/Configuration.java}{Github}]
class Configuration {
    boolean[] registers; // for pebble game
    Stack<FunctionCall> stack;

    Fun currentFunction () { ... }
    void pop() { ... }
    int getRegister() { ... }
    void freeRegister(int reg) { ... }
}
\end{codeblock}

\subsection{VarReg}
Considering the tree structure of the program, it is convenient to employ recursive traversal algorithms in the code generation.
Moreover, by itself, code generation entails only adding the handler functions for the respective tree patterns. However, this
approach needs to be integrated with the current implementation of the pebble game and it's vital
to keep track of the registers of the corresponding values. For that purpose, we add one more class:
\begin{codeblock}[VarReg class. Java implementation on \href{https://github.com/fyfsb/dcfg/blob/main/src/main/java/model/VarReg.java}{Github}]
class VarReg {
    Variable var;
    int reg;
    VarType type;
}
\end{codeblock}
The \verb+VarReg+ class is the main data structure for generating the assembly instructions.
It stores the reference to the variable, for \verb+<id>+ bindings, type for ensuring the type safety and the register
assigned from the pebble game and used in the assembly instruction.
\verb+Variable+ and \verb+VarType+ are separated. When dealing with \verb+<id>+ bindings, the respective variable is
extracted from the struct and \verb+type = var.type+. Other case is handling constants, for that the respective
type is set
\begin{itemize}
\item \verb+<CC>+ - \verb+CHAR+
\item \verb+<C>+ - \verb+INT+ or \verb+UINT+
\item \verb+<BC>+ - \verb+BOOL+
\end{itemize}
and the variable attribute is set to \verb+null+. You can find the usages of the \verb+VarReg+ class
described in the Section \ref{sec:code_gen}

\section{Code Generation}\label{sec:code_gen}
In this section, using the data structures defined so far, we introduce the classes responsible for the code generation, and describe the handler
functions for each pattern. For simplicity, classes are grouped by the context and purpose. Having almost all
methods static, these classes encapsulate nothing except the generated instruction list. We have
\subsection{ConstEvaluator}
This class generates instructions for \verb+<C>+, \verb+<BC>+ and \verb+<CC>+:
\begin{codeblock}[Constant Evaluator. Java implementation on \href{https://github.com/fyfsb/dcfg/blob/main/src/main/java/codegen/ConstantEvaluator.java}{Github}]
class ConstEvaluator {
    evaluateNumberConstant(DTE c) -> VarReg {
        assert c is <C>

        reg := Configuration.getFreeReg()
        value := parse(c)
        type := if c ends with 'u' then UINT else INT

        case type == UINT -> {
            add new instruction [ addi $reg $reg value ]
        }

        otherwise -> {
            add new instruction [ addiu $reg $reg value ]
        }

        return new VarReg(reg, type)
    }

    evaluateBooleanConstant(DTE bc) -> VarReg {
        assert bc is <BC>
        reg := Configuration.getFreeReg()
        word := bw(bc)

        value = if word == 'true' then 1 else 0
        add new instruction [ addi $reg $reg value]
        return new VarReg(reg, BOOL)
    }

    evaluateCharConstant(DTE cc) -> VarReg {
        assert cc is <CC>
        reg := Configuration.getFreeReg()
        value := parse(cc)
        add new instruction [ addi $reg $reg value ]
        return new VarReg(reg, CHAR)
    }
}
\end{codeblock}
\subsection{Expression Evaluator}
Generates instructions for arithmetic and boolean expressions, i.e. \verb+<E>+ and \verb+<BE>+.
Implementation is described later in the section.
\begin{codeblock}[ExpressionEvaluator. Java implementation on \href{https://github.com/fyfsb/dcfg/blob/main/src/main/java/codegen/ExpressionEvaluator.java}{Github}]
class ExpressionEvaluator {
    // input <E>
    evaluateArithmeticExpression(DTE e) -> VarReg { ... }

    // input <BE>
    evaluateBooleanExpression(DTE be) -> VarReg { ... }

    // Straightforward case split on 'op'
    evaluateBinaryOperation(VarReg left, DTE op, VarReg right) -> VarReg { ... }

    // input <Atom>
    evaluateAtom(DTE atom) -> VarReg { ... }
}
\end{codeblock}

\subsection{Id Evaluator}
Binds and evaluates the variables (\verb+<id>+). The method \verb+evaluateId+ gets passed the \verb+<id>+ token,
whereas \verb+bindVariableName+ binds the \verb+bw(Na)+ to the respective memory struct, and generates the code for
calculating the base address of the variable. The detailed review of the implementation can be found in the Section 2.5.7.
\begin{codeblock}[IdEvaluator. Java implementation on \href{https://github.com/fyfsb/dcfg/blob/main/src/main/java/codegen/IdEvaluator.java}{Github}]
class IdEvaluator {
    evaluateId(DTE id) -> VarReg { ... }
    bindVariableName(DTE na) -> VarReg { ... }
}
\end{codeblock}
\subsection{Memory Helper}
Contains the logic for increasing/decreasing stack and heap pointers. This class is used in the allocations and function calls.
\begin{codeblock}[MemoryHelper. Java implementation on \href{https://github.com/fyfsb/dcfg/blob/main/src/main/java/codegen/MemoryHelper.java}{Github}]
class MemoryHelper {
    increaseHeapPointer(int size) { ... }
    increaseStackPointer(int size) { ... }
}
\end{codeblock}
\subsection{Code Generator}
Gets passed the initial program, encapsulates the generated instructions, delegates tree patterns to other
evaluator classes. Directly manages function calls, conditional and loop statements.

\begin{codeblock}[Code Generator. Java implementation on \href{https://github.com/fyfsb/dcfg/blob/main/src/main/java/codegen/CodeGenerator.java}{Github}]
class CodeGenerator {
    generateCodeForFunctionCall(FunctionCall call) {
        function := call.function
        if !FunctionTable.containsFunction(function) -> { error }

        // any function other than 'main'
        if call.rds is defined -> {
            reg := Configuration.getFreeReg()
            add new instructions [
                addi $reg $0 -(size(function) - 4),
                addi $RA $reg 0
            ]
            Configuration.freeRegister(reg) // release
        }

        // body -> RSt | StS ; RSt
        case function.body.fson is <body> -> generateRSt(body.fson, call)
        otherwise -> {
            generateStS(body.fson)
            generateRSt(body.fson.bro.bro, call)
        }
    }

    generateStS(DTE sts) {
        for all st in fseq(sts) -> {
            generateSt(st)
            Configuration.freeAllRegisters()
        }
    }

    ...
}
\end{codeblock}
\newpage
As can be seen in \verb+CodeGenerator+, starting the compilation process entails creating a function call for
\verb+main+ function and forwarding it to \verb+CodeGenerator.generateCodeForFunctionCall+ method:
\begin{codeblock}
if !FunctionTable.containsFunction("main") -> { halt } // we are done

mainFunction := FunctionTable.getFunction("main")
mainCall := Configuration.createFunctionCall(mainFunction)

CodeGenerator.generateCodeForFunctionCall(mainCall)
\end{codeblock}
The method checks if the result destination is defined (for any function other than main it will be defined
and contain the pointer to the base address of the variable the result gets assigned to). In such case, the return address
register gets assigned the base address of the result destination variable. Subsequently, methods generating code for
statements are called.

\subsection{Generating for Statements}
To generate code for statements, a case split is performed based on statement types, as depicted in Figure~\ref{fig:st_types}.
Three distinct patterns are identified:
\begin{itemize}
    \item Assignments: Statements commencing with and identifier followed by an assignment operator.
    \item Loops: Statements beginning with the \verb+while+ keyword.
    \item Conditionals: Statements initiated with the \verb+if+ keyword.
\end{itemize}

\begin{figure}[h]
\centering
\begin{tikzpicture}
[font=\small,sibling distance =1.5cm, level distance=1.5cm, grow'=right, edge from parent/.style={draw=black, thick,->},
level 3/.style={level distance=1cm},
level 4/.style={level distance=3cm}]

\node [style A] (st) {St}
child {
node [style A] (id) {Id}
child {
node [style A] (eq) {=}
child { node[draw=none] (dots) {$\dots$} edge from parent[draw=none]
child { node[style B] (assignment) {generateAssignment}}
}
}
}
child {
node [style A] (while) {while}
child {
node [draw=none] (eq) {$\dots$}
child { node[draw=none] (dots) {$\dots$} edge from parent[draw=none]
child { node[style B] (loop) {generateLoop}}
}
}
}
child {
node [style A] (if) {if}
child {
node [draw=none] (eq) {$\dots$}
child { node[draw=none] (dots) {$\dots$} edge from parent[draw=none]
child {node[style B] (conditional) {generateConditional}}}
}
};

\end{tikzpicture}
\caption{Statement types}
\label{fig:st_types}
\end{figure}
\newpage
\subsection{Generating Assignments}
As illustrated in Figure~\ref{fig:gen_assignment}, the evaluation of an identifier involves storing the left value of the corresponding variable inside a register,
returned within the \verb_VarReg_ record instance \verb+verRegId+.
On the other hand, the assigned value (\verb+X+) is also evaluated, and its right value is stored in the register associated with \verb+varRegValue+.
Both instances of \verb+VarReg+ contain the types of the variables, and a verification is conducted to ensure their equality.
In case of a mismatch, a type safety error is raised; otherwise, the assignment is concluded with a store instruction.
\begin{figure}[h]
\centering
\begin{tikzpicture}
[font=\small,sibling distance =1.5cm, level distance=1.5cm, grow'=right, edge from parent/.style={draw=black, thick,->}]
\node[style A] (id) {Id}
child {
node [style A] (eq) {=} child { node[style A] (x) {X}}
};

\node[style B] at (0, -1.5) (evaluateId) {\verb+evaluateId+};
\node[style A] at (0, -3) (varRegId) {\verb+varRegId+};
\node[style B] at (3, -1.5) (getValue) {\verb+getValue+};
\node[style A] at (3, -3) (varRegValue) {\verb+varRegValue+};
\node[style B, minimum width=3cm] at (1.5, -4.5) (check) {\verb+checkTypesSame+};
\node[style D] at (4.5, -4.5) (error) {Error};
\node[style A] at (1.5, -6) (instr) {add instr: \verb+sw $varRegValue.reg $varRegId.reg 0+};

\draw[thick, ->] (id) -- (evaluateId);
\draw[thick, ->] (x) -- (getValue);
\draw[thick, ->] (evaluateId) -- (varRegId);
\draw[thick, ->] (getValue) -- (varRegValue);
\draw[thick, ->] (varRegId) -- (check);
\draw[thick, ->] (varRegValue) -- (check);
\draw[thick, ->] (check) -- node[midway, below] {\tiny false} (error);
\draw[thick, ->] (check) -- node[midway, right] {\tiny true} (instr);
\end{tikzpicture}
\caption{Generate Assignment}
\label{fig:gen_assignment}
\end{figure}

The evaluation of the identifier, as shown in the listing below, receives the
\verb+<id>+ token and a boolean parameter \verb+lv+, indicating the left value.
When \verb+lv=false+, the right value of the identifier is needed.
Upon computing the base address, an additional dereference is appended to the instruction list.
Otherwise, no extra dereference is added. Regardless of the \verb+lv+ parameter, if identifier
derives the usage of 'address-of' operator, extra dereference is omitted. Base address of the variable
is bound by the call of \verb+bindVariable+ from class \verb+IdEvaluator+.

\begin{figure}[h]
\centering
\begin{codeblock}[evaluateId. Java implementation on \href{https://github.com/fyfsb/dcfg/blob/main/src/main/java/codegen/IdEvaluator.java}{Github}]
input: DTE id, bool lv (checks if additional dereference is required)
output: VarReg result - bound identificator

evaluateId(DTE id, bool lv) -> VarReg {
    res := bindVariable(id)
    if (id.children is not <id>& and !lv) {
        add new instruction [lw $res.reg $res.reg 0] // dereference
    }
    return res
}
\end{codeblock}
\label{fig:evalId}
\end{figure}

\newpage
Binding the variable based on the pattern:
\begin{itemize}
\item \verb+<Na>+: Name is bound either from the global memory, or from the current function's struct.
\item \verb+<id>.<Na>+: Structure's identifier is evaluated recursively. Then a search of the component
with name \verb+bw(Na)+ is performed and the register storing the base address of the struct is increased by
the displacement of the component.
\item \verb+<id>[...+: Array's identifier is evaluated recursively. The value inside the square brackets is calculated,
and the register storing the base address of the array is increased by the standard displacement of the array's element:
\[ba(arr[i]) = ba(arr) + size(type(arr[i])) \cdot i\]
\item \verb+<id>*+: Pointer access entails additional dereferencing of the recursively evaluated pointer's identifier.
\item \verb+<id>&+: As mentioned above, only name is bound, no additional instructions get added.
\end{itemize}

\begin{codeblock}[bindVariable. Java implementation on \href{https://github.com/fyfsb/dcfg/blob/main/src/main/java/codegen/IdEvaluator.java}{Github}]
input: DTE id
output: VarReg result

bindVariable(DTE id) -> VarReg {
    case id.children is <Na> -> {
        return bindName(id.fson)
    }

    case id.children is <id>.<Na> -> {
        structVar := evaluateId(id.fson)

        compName := bw(id.fson.bro.bro)
        structComp := getComponent(structVar.type, compName)
        j := structVar.j and d := displ(structComp, structVar.var)
        add new instruction [addi $j $j d]
        return structComp
    }

    case id.children is <id>[... -> return bindArrElement(id)
    case id.children is <id>* -> {
        res := evaluateId(id.fson)
        add new instruction [lw $res.reg $res.reg 0] // deref
        return res
    }
    case id.children is <id>& -> {
        return bindName(id.fson)
    }
}
\end{codeblock}

To bind the name, we first access the current function, from the \verb+Configuration+ class. Local variable
is then checked in the function's struct. If found, name is then bound locally, otherwise, the variable is searched
in the global memory's struct.
\begin{codeblock}[bindName. Java implementation on \href{https://github.com/fyfsb/dcfg/blob/main/src/main/java/codegen/IdEvaluator.java}{Github}]
input: DTE na - name tree element
output: VarReg res - bound variable

bindName(DTE na) -> VarReg {
    cf := Configuration.currentFunction
    cfms := cf.memoryStruct

    case bw(na) is in cfms -> { // bind local variable from the cf
        comp := gm.getComponent(bw(na))
        reg := Configuration.getFreeReg() // pebble game
        add new instruction [addi $reg $spt displ(comp, cf) - size(cf)]
        return new VarReg(reg, comp)
    }
    otherwise -> { // bind global variable
        memory := MemoryTable.gm
        comp := gm.getComponent(bw(na)) // Variable or null
        if comp is null -> { error } // not in gm

        reg := Configuration.getFreeReg() // pebble game
        add new instruction [addi $reg $bpt displ(comp, memory)]
        return new VarReg(reg, comp)
    }
}
\end{codeblock}

For the assigned value, we differentiate among arithmetic and boolean expressions, character constants,
heap allocations and function calls. Any other pattern is considered a grammar error.
\begin{codeblock}[getValue. Java implementation on \href{https://github.com/fyfsb/dcfg/blob/main/src/main/java/codegen/CodeGenerator.java}{Github}]
input: DTE x - tree element representing assigned value
output: VarReg varRegValue - bound value

getValue(DTE x) -> VarReg {
    case x.children is <E> -> return evaluateExpr(x.fson)
    case x.children is <BE> -> return evaluateBE(x.fson)
    case x.children is <CC> -> return evaluateCharConst(x.fson)
    case x.children is new <Na>* -> return evaluateAlloc(x.fson.bro)
    case x.children is <Na>(... -> return funCall(x)
    otherwise -> error // grammar error
}
\end{codeblock}

\newpage
For the arithmetic expressions, we rely on the fact, that token \verb+<E>+ has at most 3 direct children.
\begin{itemize}
\item \verb+<E>+ has 3 children: it is either a binary operation, or a factor inside parenthesis \verb+(F)+.
In case of the latter, inner factor is evaluated recursively, otherwise operands and the operator are extracted from the
parent tree element, and the instruction corresponding to the respective operation (\verb_+,-,*,/_) and the type (\verb+INT, UINT+)
is added to the instruction list.
\item \verb+<E>+ has 2 children: Single occurrence of such pattern is the case with unary minus on the factor. Factor is
evaluated recursively, and the value inside the register is negated.
\item \verb+<E>+ has 1 child: If \verb+<T>+ or \verb+<F>+ appear, we continue traversing the children, case of the
identifier is handled by the aforementioned \verb+evaluateId+, arithmetic constants are parsed with \verb+ConstEvaluator+ class.
\end{itemize}
\begin{codeblock}[evaluateExpr. Java implementation on \href{https://github.com/fyfsb/dcfg/blob/main/src/main/java/codegen/ExpressionEvaluator.java}{Github}]
input: DTE e - expression tree element
output: VarReg res - bound value and register of the expression

evaluateExpr(DTE e) -> VarReg {
    case size(e.children) is 3 -> {
        case e.children is (<F>) -> evaluateExpr(e.fson.bro)
        otherwise -> {
            firstOperand := e.fson
            operator := firstOperand.bro
            secondOperand := operator.bro
            evalBinaryOperation(firstOperand, operator, secondOperand)
        }
    }
    case size(e.children) is 2 -> {
        f := evaluateExpr(e.fson.bro)
        add new instruction [ sub $f.reg 0 $f.reg ]
        return f
    }
    case size(e.children) is 1 -> {
        case e.fson is <T> or <F> -> evaluateExpr(e.fson)
        case e.fson is <id> -> evaluateId(e.fson, false)
        case e.fson is <C> -> evaluateConst(e.fson)
    }
    otherwise -> error
}
\end{codeblock}

\newpage
Allocations entail storing the heap pointer's current value inside the register bound to the first identifier of
the assignment statement, after which \verb+MemoryHelper+'s method for increasing the heap pointer is utilized. Size of
the increase, is the size of the pointer's target type.
\begin{codeblock}[evaluateAlloc. Java implementation on \href{https://github.com/fyfsb/dcfg/blob/main/src/main/java/codegen/CodeGenerator.java}{Github}]
input: DTE x - tree element deriving new <Na>*
output: new heap entry gets allocated, heap pointer increased.

evaluateAlloc(DTE x) {
    add new instruction [ sw $hpt $varRegId.reg 0 ]
    MemoryHelper.increaseHPT(size(varRegId.var))
}
\end{codeblock}
To create the function call, we first check the existence of the function with name \verb+bw(Na)+ in the
function table. The stack pointer is then increased. Subsequently, if present, the parameter values are set.
Next step is to initialize local variables, number of the memory words occupied is obtained and the respective range on the
stack is filled with zeros. Result destination stack is initialized with the base address of the identifier.
Jump and link instruction to the called function's label is added to the instruction list, after which the \verb+CodeGenerator+'s
method \verb+generateCodeForFunctionCall+ is called.
\begin{codeblock}[funCall. Java implementation on \href{https://github.com/fyfsb/dcfg/blob/main/src/main/java/codegen/CodeGenerator.java}{Github}]
input: DTE x - tree element deriving <Na> (<PaS>) {...
output: new function call gets created, stack pointer increased

funCall(x) {
    name := x.fson
    function := FunTable.getFunction(name)
    MemoryHelper.increaseSPT(size(function))

    if x.nthSon(3) is <PaS> {
        funMemory = function.getMemoryStruct()
        for all i = 0,1,... size(x.nthSon(3)) - 1 -> {
            VarReg p := evalPa(x.nthSon(3)[i]) // pa -> <E>|<BE>|<CC>
            Variable strParam := funMemory.at(i)

            imm := -size(function) + displ(function, strParam)
            add new instruction [ sw $p.reg $spt imm ]
            Configuration.freeRegister(p.reg) // rm node from pebble
        }
    }

    mwo := size(function.localVariables)
    reg1, reg2 := Configuration.getFreeRegisters()
    firstLocal := function.localVariables[0]

    add new instruction [ add $reg1 $spt displ(function, firstLocal) ]
    add new instruction [ addi $reg2 $0 mwo ]
    add new instruction [ zero(reg1, reg2) ]
    Configuration.freeRegisters(reg1, reg2)

    initRDS()
    add new instruction [ jal $function.label ]
    call := createFunctionCall(function)
    generateCodeForCall(call)
}

\end{codeblock}
\subsubsection{Evaluating the Return Statement}

Returned result is an expression, which is evaluated. Subsequently, the result is assigned to the
identifier in the assignment by dereferencing the \verb+rds+ pointer. Function then is ready either to
return or to halt (if name="main"). In case of the former, stack pointer is decreased by the size of the function,
return address is obtained and the jump performed.
\begin{codeblock}[generateRSt. Java implementation on \href{https://github.com/fyfsb/dcfg/blob/main/src/main/java/codegen/CodeGenerator.java}{Github}]
generateRSt(DTE rSt, FunctionCall call) {

    // RSt = return X, X = <E>|<BE>|<CC>
    res := evaluateResult(body.fson.bro)
    rdsReg := bind(function.rds)
    add new instruction [ sw $res.reg $rdsReg 0]
    if Configuration.currentFunction is "main" -> { HALT program }

    add new instruction [ lw $1 $spt -(size(function) - 4)]
    add new instruction [ addi $spt $spt -(size(function) - 4)]
    add new instruction [ jr $1 ]
}
\end{codeblock}

\newpage
\subsection{Loops}
Main concern with loops and conditionals is calculating the jump distance. Upon the evaluation of the boolean expression,
following branch jump should contain the size of the instructions of the body, which is not yet evaluated. Workaround is
to first generate the instructions for the body of the loop, store the size and then insert the respective jump instructions
before and after the body. For that purpose, indices before and after the \verb+StS+ instructions in the list are also
recorded. Other than that, integration is straightforward, as depicted in Figure~\ref{fig:while_st}.
\begin{figure}[h]
\centering
\begin{tikzpicture}
[font=\small,sibling distance =1.5cm, level distance=1.5cm, grow'=right, edge from parent/.style={draw=black, thick,->}]
\node[style A] (while) {\verb+while+}
child {
node[style A] (be) {BE}
child {
node[style A] (lcb) {\{}
child {
node[style A] (sts) {StS} child { node[style A] (rcb) {\}}}
}
}
};

\node[style B] at (1.5, -2) (evalBE) {\verb+evalBE+};
\node[style B] at (4.5, -2) (fseq) {\verb+fseq+};
\node[style B] at (7, -2) (evalSt) {\verb+evalSt+};
\node[draw=none] at (7.5, -2.7) (afterSts) {save afterSts};
\node[style A] at (1.5, -4) (bres) {\verb+bRes: VarReg+};
\node[style A] at (2, -6) (finstr) {instr: \verb_beqz $bRes.reg (bodySize + 2)_};
\node[style A] at (2, -8) (sinstr) {instr: \verb_blez $0 - (BESize + bodySize + 1)_};
\node[style A, minimum height=5cm, minimum width=5cm] at (10, -6) (box) {};
\node[draw=none] at (10,-3.7) (pinstr) {program instructions};
\node[draw=none] at (10, -4.25) (dots) {$\dots$};
\node[style A] at (10, -5) (beBody) {\verb_BE_ evaluation};
\node[draw=none] at (9, -6) (fjump) {};
\node[draw=none] at (9, -8) (sjump) {};
\node[style A] at (10, -7) (whileBody) {\verb_while_ body evaluation};

\draw[thick, ->] (be) -- node[midway,left] {\small save beforeBE} (evalBE);
\draw[thick, ->] (sts) -- node[midway, right] {\small save beforeStS} (fseq);
\draw[thick, ->] (fseq) -- node[midway, below] {\tiny for each} (evalSt);
\draw[thick, ->] (evalBE) -- node[midway, left] {\small save afterBE} (bres);
\draw[thick, ->] (finstr) -- (fjump);
\draw[thick, ->] (sinstr) -- (sjump);
\end{tikzpicture}
\caption{Evaluating loops}
\label{fig:while_st}
\end{figure}

\newpage
\subsection{If Statements}
Evaluation of the \verb+if+ statements has similarities with loops. Since there is a branch jump of the size
of the `if` body, the inner \verb+StS+ is evaluated first and the size of the body, index before the statement sequence
are recorded. Subsequently, branch instruction is inserted in the corresponding position in the list.
\begin{figure}[h]
\begin{tikzpicture}
[font=\small,sibling distance =1.5cm, level distance=1.5cm, grow'=right, edge from parent/.style={draw=black, thick,->}]
\node[style A] (while) {\verb+if+}
child {
node[style A] (be) {BE}
child {
node[style A] (lcb) {\{}
child {
node[style A] (sts) {StS} child { node[style A] (rcb) {\}}}
}
}
};

\node[style B] at (1.5, -2) (evalBE) {\verb+evalBE+};
\node[style B] at (4.5, -2) (fseq) {\verb+fseq+};
\node[style B] at (7, -2) (evalSt) {\verb+evalSt+};
\node[draw=none] at (7.5, -2.7) (afterSts) {save afterSts};
\node[style A] at (1.5, -4) (bres) {\verb+bRes: VarReg+};
\node[style A] at (2, -6) (finstr) {instr: \verb_beqz $bRes.reg (ifSize + 1)_};
\node[style A, minimum height=5cm, minimum width=5cm] at (10, -6) (box) {};
\node[draw=none] at (10,-3.7) (pinstr) {program instructions};
\node[draw=none] at (10, -4.25) (dots) {$\dots$};
\node[style A] at (10, -5) (beBody) {\verb_BE_ evaluation};
\node[draw=none] at (9, -6) (fjump) {};
\node[draw=none] at (9, -8) (sjump) {};
\node[style A] at (10, -7) (whileBody) {\verb_if_ body evaluation};

\draw[thick, ->] (be) -- (evalBE);
\draw[thick, ->] (sts) -- node[midway, right] {\small save beforeStS} (fseq);
\draw[thick, ->] (fseq) -- node[midway, below] {\tiny for each} (evalSt);
\draw[thick, ->] (evalBE) -- (bres);
\draw[thick, ->] (finstr) -- (fjump);
\end{tikzpicture}
\caption{If statement evaluation}
\label{fig:if_st}
\end{figure}

\newpage
\subsection{If/Else Statements}
Only difference from the regular \verb+if+ statements is that an unconditional jump is added in the end of the
if body, to skip the else part. Given adjustments, initial branch jump's distance gets incremented. Other than that,
realization is straightforward.
\begin{center}
\begin{tikzpicture}
[font=\small,sibling distance =1.5cm, level distance=1.5cm, grow'=right, edge from parent/.style={draw=black, thick,->}]
\node[style A] (if) {\verb+if+}
child {
node[style A] (be) {BE}
child {
node[style A] (lcb) {\{} child {
node[style A] (ifsts) {StS} child {node[style A] (rcb) {\}} child {
node[style A] (else) {else} child {node[style A] (lcb1) {\{} child {
node[style A] (elsests) {StS} child {node[style A] (rcb1) {\}}}
}}
}}
}
}
};

\node[style B] at (1.5, -2) (evalBE) {\verb+evalBE+};
\node[style B] at (4.5, -2) (fseq) {\verb+fseq+};
\node[style B] at (7, -2) (evalSt) {\verb+evalSt+};
\node[draw=none] at (7.5, -2.7) (afterIfPart) {save afterIfPart};

\node[style B] at (10, -2) (fseq2) {\verb+fseq+};
\node[style B] at (11.5, -2) (evalSt2) {\verb+evalSt+};
\node[draw=none] at (12, -2.7) (afterElsePart) {save afterElsePart};


\node[style A] at (1.5, -4) (bres) {\verb+bRes: VarReg+};
\node[style A] at (2, -6) (finstr) {instr: \verb_beqz $bRes.reg (ifSize + 2)_};
\node[style A] at (2, -8) (sinstr) {instr: \verb_blez $0 - (elseSize + 1)_};
\node[style A, minimum height=5cm, minimum width=5cm] at (10, -6) (box) {};


\node[draw=none] at (10,-3.7) (pinstr) {program instructions};
\node[draw=none] at (10, -4.25) (dots) {$\dots$};
\node[style A] at (10, -5) (beBody) {\verb_BE_ evaluation};
\node[draw=none] at (9, -5.5) (fjump) {};
\node[draw=none] at (9, -6.5) (sjump) {};
\node[style A] at (10, -6) (ifBody) {\verb_if_ body evaluation};
\node[style A] at (10, -7) (elseBody) {\verb_else_ body evaluation};

\draw[thick, ->] (be) -- (evalBE);
\draw[thick, ->] (ifsts) -- node[midway, right] {\small save beforeIfPart} (fseq);
\draw[thick, ->] (elsests) -- node[midway, right] {\small save beforeElsePart} (fseq2);
\draw[thick, ->] (fseq) -- node[midway, below] {\tiny for each} (evalSt);
\draw[thick, ->] (evalBE) -- (bres);
\draw[thick, ->] (finstr) -- (fjump);
\draw[thick, ->] (sinstr) -- (sjump);
\end{tikzpicture}
\end{center}

\section{Translated programs}\label{sec:translated_programs}
In this section, we provide the translations of some programs written in C0 (Sample programs are copied from the system architecture book exercises on code generation \cite{sysbook}).
Listed programs are in c0 syntax, with the applied pre-processing (spaces and tabulations trimmed, dedicated `EOF` symbol appended),
whereas the compiled versions have jumps and links to the function label instead of the relative address, untranslated macros, the syscalls
used implicitly (e.g. HALT) and the comments for readability. With this in account, we delegate the complete translation of the program to the assembler.\\~\\
First example shows the program with assignments and conditional statements:
\begin{codeblock}
int x;
int main(){
x = 3;
x = x + 1;
if x>0 {x = 1} else {x = 2};
return 0
}~
\end{codeblock}
With it's respective translation:
\begin{codeblock}
_main:
addi $1 $28 0
addi $2 $2 3
sw $2 $1 0
addi $1 $28 0
addi $2 $28 0
lw $2 $2 0
addi $3 $3 1
add $2 $2 $3
sw $2 $1 0
addi $1 $28 0
lw $1 $1 0
addi $2 $2 0
sgt 1 1 2
beqz 1 4
addi $2 $28 0
addi $3 $3 1
sw $3 $2 0
addi $1 $1 0
HALT
\end{codeblock}

The program with a loop:
\begin{codeblock}
int n;
int res;
int main(){
n = 32768;
res = 0;
while n>0 {
res = res + n;
n = n - 1
};
return 0
}~
\end{codeblock}
Translated version:
\begin{codeblock}
_main:
addi $1 $28 0
addi $2 $2 32768
sw $2 $1 0
addi $1 $28 4
addi $2 $2 0
sw $2 $1 0
addi $1 $28 0
lw $1 $1 0
addi $2 $2 0
sgt 1 1 2
beqz 1 15
addi $2 $28 4
addi $3 $28 4
lw $3 $3 0
addi $4 $28 0
lw $4 $4 0
add $3 $3 $4
sw $3 $2 0
addi $1 $28 0
addi $2 $28 0
lw $2 $2 0
addi $3 $3 1
sub $2 $2 $3
sw $2 $1 0
blez 0 -18
addi $1 $1 0
HALT
\end{codeblock}
The program calculating the nth Fibonacci number:
\begin{codeblock}
int x;

int fib(int n){
int res;
int first;
int second;

if n<2 {res = n} else {
first = fib(n - 2);
second = fib(n - 1);
res = first + second
};

return res
};

int main(){
x=fib(31);
return 0
}~
\end{codeblock}
Translated version:
\begin{codeblock}
_main:
addi $1 $28 0
addi $2 $29 16 # start of increasing spt
subi $2 $2 32
blez 2 4
macro: gpr(1) = x
sysc
addi $29 $29 20 # end of increasing spt
add $2 $29 $-12
addi $3 $0 12
macro: zero($2, $3)
jal _fib
addi $1 $1 0
HALT

_fib:
addi $2 $0 -20
sw $31 $2 0
addi $2 $29 -16
lw $2 $2 0
addi $3 $3 2
slt $2 $2 $3
beqz 2 6
addi $3 $29 -12
addi $4 $29 -16
lw $4 $4 0
sw $4 $3 0
beq 0 30
addi $2 $29 -8
addi $3 $29 16 # start of increasing spt
subi $3 $3 32
blez 3 4
macro: gpr(1) = x
sysc
addi $29 $29 20 # end of increasing spt
add $3 $29 $-12
addi $4 $0 12
macro: zero($3, $4)
jal _fib
addi $3 $29 -4
addi $4 $29 16 # start of increasing spt
subi $4 $4 32
blez 4 4
macro: gpr(1) = x
sysc
addi $29 $29 20 # end of increasing spt
add $4 $29 $-12
addi $5 $0 12
macro: zero($4, $5)
jal _fib
addi $4 $29 -12
addi $5 $29 -8
lw $5 $5 0
addi $6 $29 -4
lw $6 $6 0
add $5 $5 $6
sw $5 $4 0
addi $4 $29 -12
lw $4 $4 0
sw $4 $1 0 # start of return from fib
lw $1 $29 -20
addi $29 $29 -20
jr 1 # end of return from fib
\end{codeblock}
\newpage
The program calculating powers, using the defined struct:
\begin{codeblock}
typedef struct {int exp; int val} powst;
int x;

int main(){
x=pow(2, 2);
return x
};

int pow(int base, int n){
powst p;
p = new powst*;
p.exp = 0;
p.val = 1;
while p.exp<n {
p.exp = p.exp + 1;
p.val = p.val*base
};
return p.val
}~
\end{codeblock}
Translation:
\begin{codeblock}
_main:
addi $1 $28 0
addi $2 $29 16 # start of increasing spt
subi $2 $2 32
blez 2 4
macro: gpr(1) = x
sysc
addi $29 $29 20 # end of increasing spt
addi $2 $2 2
sw $2 $29 -16
addi $2 $2 2
sw $2 $29 -12
add $2 $29 $-8
addi $3 $0 8
macro: zero($2, $3)
jal _pow
addi $1 $28 0
lw $1 $1 0
HALT

_pow:
addi $2 $0 -20
sw $31 $2 0
addi $2 $29 -8
sw $30 $2 0
# INCREASING HEAP POINTER
addi $30 $30 8 # start of increasing hpt
subi $1 $30 31
bltz 1 4
macro: gpr(1) = x
sysc
addi $1 $30 -8
addi $2 $0 2
zero(1, 2) # end of increasing hpt
addi $2 $29 -8
addi $2 $2 0
addi $3 $3 0
sw $3 $2 0
addi $2 $29 -8
addi $2 $2 4
addi $3 $3 1
sw $3 $2 0
addi $2 $29 -8
addi $2 $2 0
lw $2 $2 0
addi $3 $29 -12
lw $3 $3 0
slt $2 $2 $3
beqz 2 19
addi $3 $29 -8
addi $3 $3 0
addi $4 $29 -8
addi $4 $4 0
lw $4 $4 0
addi $5 $5 1
add $4 $4 $5
sw $4 $3 0
addi $2 $29 -8
addi $2 $2 4
addi $3 $29 -8
addi $3 $3 4
lw $3 $3 0
addi $4 $29 -16
lw $4 $4 0
macro: mul($3, $3, $4)
sw $3 $2 0
blez 0 -24
addi $2 $29 -8
addi $2 $2 4
lw $2 $2 0
sw $2 $1 0 # start of return from pow
lw $1 $29 -20
addi $29 $29 -20
jr 1 # end of return from pow
\end{codeblock}
\newpage
Linked lists:
\begin{codeblock}
typedef LEL* u;
typedef struct {int content; u next} LEL;
u first;
u last;
int n;
int main(){
n = 200;
first = new LEL*;
last = first;
n = n -1;
while n>0 {
last*.next = new LEL*;
last = last*.next;
n = n - 1
};
return 0
}~
\end{codeblock}
Translation:
\begin{codeblock}
_main:
addi $1 $28 8
addi $2 $2 200
sw $2 $1 0
addi $1 $28 0
sw $30 $1 0
# INCREASING HEAP POINTER
addi $30 $30 4 # start of increasing hpt
subi $1 $30 31
bltz 1 4
macro: gpr(1) = x
sysc
addi $1 $30 -4
addi $2 $0 1
zero(1, 2) # end of increasing hpt
addi $1 $28 4
addi $2 $28 0
lw $2 $2 0
sw $2 $1 0
addi $1 $28 8
addi $2 $28 8
lw $2 $2 0
addi $3 $3 1
sub $2 $2 $3
sw $2 $1 0
addi $1 $28 8
lw $1 $1 0
addi $2 $2 0
sgt 1 1 2
beqz 1 27
addi $2 $28 4
lw $2 $2 0
addi $2 $2 4
sw $30 $2 0
# INCREASING HEAP POINTER
addi $30 $30 4 # start of increasing hpt
subi $1 $30 31
bltz 1 4
macro: gpr(1) = x
sysc
addi $1 $30 -4
addi $2 $0 1
zero(1, 2) # end of increasing hpt
addi $1 $28 4
addi $2 $28 4
lw $2 $2 0
addi $2 $2 4
lw $2 $2 0
sw $2 $1 0
addi $1 $28 8
addi $2 $28 8
lw $2 $2 0
addi $3 $3 1
sub $2 $2 $3
sw $2 $1 0
blez 0 -30
addi $1 $1 0
HALT
\end{codeblock}